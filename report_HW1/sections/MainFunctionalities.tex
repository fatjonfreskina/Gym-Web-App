\section{Main Functionalities}
%What are the main functionalities of the web app? what services does it offer and how it is organized?

The Web Application is designed to allow users to subscribe courses, manage the attendance and the logistics part underneath a gym (i.e. substitution of a trainer, new courses etc..).\\
The website is divided into 4 areas:

\begin{itemize}
	\item Public Area: this area is available to unregistered users. In this area a User can :
	\begin{itemize}
		\item see general information about the Gym.
		\item know who is the staff involved in the gym, in particular every trainer and any course he or she teaches.
		\item understand the courses offered by the Gym and type of Subscription offered with relative prices.
		\item know the calendar and hours of the Gym as well as the location and contact information.
		\item see some pictures of the equipment available at the gym.
	\end{itemize}	 
	In all of those pages a User can go the Authentication page and also get to the Registration page. 
	\item Trainee Area: this area is available to registered users with the role "Trainee". It includes the following pages:
	\begin{itemize}
		\item Personal info: it is used to show personal data and informations, and it offers the possibility either to change password or change avatar.
		\item Reservation management: the trainee will be provided with the lists of all the active subscriptions he or she has, with relative expiration dates. In this page he/she can see the lesson schedule and book the lessons for the current week. It's also possible to change week and book for the coming weeks.
	\end{itemize}
	\item Trainer Area: this area is available to registered users with the role "Trainer". It includes the following pages:
	\begin{itemize}
		\item Link to personal info: it is used to modify or add personal data, such as Medical Certificate
		\item Courses Management: it is used to see the lectures that the user has to hold. In this section it is possible to see how many people are enrolled in each course as well as how many lectures the trainer has already held for that particular course.
		\item Presence Management: it is used to see if someone that has booked a lesson is present or not, and to add/remove reservations manually.
	\end{itemize}
	\item Secretary Area: this area is available to registered users with the role "Secretary".
	It includes the following pages:
	\begin{itemize}
		\item Account management: possibility of updating some users' personal info, modifying roles (create new Secretary/Trainer).
		\item Courses management: possibility of adding new courses and set the respective instructors. It is also possible, if a coach gets sick or can't teach at his course, to define a substitute.
		\item Subscriptions management for users.
	\end{itemize}
\end{itemize}
